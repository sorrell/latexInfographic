% Infographic, comparing two layers
% Author: Nick Sorrell
% Based on diagram from Agostino De Marco, Marco Miani and Pascal Seppecher.
% How to use this document:
%	Below, you will see a lot of \newcommands.  Fill these values with your
%	own.  There is a Title, Image, and Label section that you will need to
%	change to make your own infographic.  You may notice that your image
%	is too small, or too large - no problem, just go to that layer, find
%	your node and increase/decrease the scale. Notice that each \node
%	ends with a ;. And that a \fill only needs one ; for the whole block
%	of code.

% Based on diagram from Agostino De Marco from diagram byMarco Miani and Pascal Seppecher.
\documentclass{article}
\usepackage{tikz}
\usepackage{verbatim}
\usepackage[active,tightpage]{preview}
\PreviewEnvironment{tikzpicture}
\setlength\PreviewBorder{5pt}%
%%%>
\usetikzlibrary{positioning}
\newcommand\sserifbf[1]{\textsf{{\bfseries{#1}}}}
\newcommand{\yslant}{0.5}
\newcommand{\xslant}{-0.6}

%%%%%%%%%%%%%%%%%%%%%%%%%%%%%%%%%%%%%%%%
% Fill title values with these commands
%%%%%%%%%%%%%%%%%%%%%%%%%%%%%%%%%%%%%%%%
\newcommand{\TLTitle}{\Large{\textbf{Cincinnati}}}
\newcommand{\BLTitle}{\Large{\textbf{Ohio}}}

%%%%%%%%%%%%%%%%%%%%%%%%%%%%%%%%%%%%%%%%
% Fill image locations with these commands
%%%%%%%%%%%%%%%%%%%%%%%%%%%%%%%%%%%%%%%%
\newcommand{\TLPicOne}	{images/bwCin.png}	%TopLayer Pic 1
\newcommand{\TLPicTwo}	{images/bizCin.png}
\newcommand{\TLPicThree}{images/capCin.png}
\newcommand{\TLPicFour}	{images/poorCinI.png}
\newcommand{\TLPicFive}	{images/downCin.png}
\newcommand{\BLPicOne}	{images/bwOhio.png}	%BottomLayer Pic 1
\newcommand{\BLPicTwo}	{images/bizOhio.png}
\newcommand{\BLPicThree}{images/capOhio.png}
\newcommand{\BLPicFour}	{images/poorOhioI.png}
\newcommand{\BLPicFive}	{images/upOH.png}

%%%%%%%%%%%%%%%%%%%%%%%%%%%%%%%%%%%%%%%%
% Fill label values with these commands
%%%%%%%%%%%%%%%%%%%%%%%%%%%%%%%%%%%%%%%%
\newcommand{\TLTextOne}		{\sserifbf{Population} \\ \sserifbf{\small (Black vs. White)} \\ \sserifbf{2010}}
\newcommand{\TLTextTwo}		{\sserifbf{Black Owned}\\ \sserifbf{Businesses} \\ \sserifbf{2007}}
\newcommand{\TLTextThree}	{\sserifbf{Bachelor's Degrees}\\ \sserifbf{2006-2010}}
\newcommand{\TLTextFour}	{\sserifbf{Persons Below}\\ \sserifbf{Poverty Line}  \\ \sserifbf{2006-2010}}
\newcommand{\TLTextFive}	{\sserifbf{Population Change} \\ \sserifbf{2000-2010}}
\newcommand{\BLTextOne}		{}
\newcommand{\BLTextTwo}		{}
\newcommand{\BLTextThree}	{}
\newcommand{\BLTextFour}	{}
\newcommand{\BLTextFive}	{}

\begin{document}
\begin{tikzpicture}[scale=1.1,every node/.style={minimum size=1cm},on grid]

	%%%%%%%%%%%%%%%%% 
	% Bottom Layer
	%%%%%%%%%%%%%%%%%
	\begin{scope}[
		yshift=-120,
		every node/.append style={yslant=\yslant,xslant=\xslant},
		yslant=\yslant,xslant=\xslant
		] 
		
	% The lower frame:
		\fill[white,fill opacity=.70] 	(-1.3,0) rectangle (11,4.8);		%Optional, needed for my infographic
		\draw[black, dashed, thick] (-1.3,0) rectangle (11,4.8); 
		
	% Dots:
		\draw[fill=red]  
			(9.7,4.5)	circle (.1)		% 5th dot
			(7.5,4.5) 	circle (.1) 	% 4th dot
			(5,4.5) 	circle (.1) 	% 3rd dot
			(2,4.5) 	circle (.1) 	% 2nd dot
			(-0.5,4.5) 	circle (.1)  	% 1st dot
		;
		
	% Icons: add fill=white if needed for graphics
		\node[anchor=south,inner sep=0,xshift=-20pt,yshift=10pt,scale=1.05] at (0.3,1)
			{\includegraphics[width=2cm]{\BLPicOne}};
		\node[anchor=south,inner sep=0,xshift=0pt,yshift=8pt,scale=1.3] at (2.3,1)
			{\includegraphics[width=2cm]{\BLPicTwo}};
		\node[anchor=south,inner sep=0,xshift=-5pt,yshift=8pt,,scale=1.4] at (5.1,0.8)
			{\includegraphics[width=2cm]{\BLPicThree}};
		\node[anchor=south,inner sep=0,xshift=20pt,yshift=8pt,,scale=0.65] at (6.8,1)
			{\includegraphics[width=2cm]{\BLPicFour}};
		\node[anchor=south,inner sep=0,xshift=20pt,yshift=8pt,,scale=1.2] at (9,1)
			{\includegraphics[width=2.5cm]{\BLPicFive}};
	
	% Labels:		
		\fill[black]
		 	(-0.2,-0.1) 	node[above=-2pt, scale=1.1] 
				{\BLTitle}
			(9.7,4.5)		node[below right,,xshift=-20pt,yshift=-5pt,scale=.9,text width=2.5cm,align=left]
				{\BLTextFive}
			(7.3,1.45) 		node[below right,,xshift=-20pt,yshift=-5pt,scale=.9,text width=2.5cm,align=left]
				{\BLTextFour}
			(5.0,1.2) 		node[below right,xshift=-20pt,scale=.9,text width=2cm,align=left]
				{\BLTextThree}
			(2.1,1.3) 		node[below right,xshift=-10pt,scale=.9,text width=2cm,align=left]
				{\BLTextTwo}
			(0.2,1.5) 		node[below right,xshift=-20pt,yshift=-5pt,scale=1.1,text width=2.5cm,align=left]
				{\BLTextOne}
		;
	\end{scope}
	
	%%%%%%%%%%%%%%%%%%%%%%%%%%%%%%%%%%%%%%%%%%%%%%%%%%%%%%%
	% Vertical lines for linking red dots on the 2 layers
	%%%%%%%%%%%%%%%%%%%%%%%%%%%%%%%%%%%%%%%%%%%%%%%%%%%%%%%
	\draw[thick](9,6.5)		to (7,3.86);		%5th line to fifth dots
	\draw[thick](6.3,5.1) 	to (4.8,2.76);		%4th line to fourth dots
	\draw[thick](3.8,4) 	to (2.3,1.5);		%3rd line to third dots
	\draw[thick](0.8,2.4) 	to (-0.7,-0.03);  	%2nd line to second dots
	\draw[thick](-2.2,0.9) 	to (-3.2,-1.25);	%1st line to first dots
	
	%%%%%%%%%%%%%%%%% 
	% Top Layer
	%%%%%%%%%%%%%%%%%
	\begin{scope}[
		yshift=0,
		every node/.append style={yslant=\yslant,xslant=\xslant},
		yslant=\yslant,xslant=\xslant
		]
		
	% The upper frame:
		\fill[white,fill opacity=.70] 	(-3.1,0) rectangle (12,6); % Opacity
		\draw[black, dashed, thick] 	(-3.1,0) rectangle (12,6); 
		
	% Red dots:
		\draw [fill=red]
			(10.3,2)	circle (.1)		% 5th dot
			(7.5,2) 	circle (.1) 	% 4th dot
			(5,2) 		circle (.1) 	% 3rd dot
			(2,2) 		circle (.1) 	% 2nd dot
			(-1,2) 		circle (.1)	 	% 1st dot
		;
		
	% Icons:
		\node[anchor=south,inner sep=0,xshift=-20pt,yshift=10pt] at (-0.5,2)
			{\includegraphics[width=2cm]{\TLPicOne}};
		\node[anchor=south,inner sep=0,xshift=0pt,yshift=8pt,scale=1.3] at (2,2)
			{\includegraphics[width=2.5cm]{\TLPicTwo}};
		\node[anchor=south,inner sep=0,xshift=-5pt,yshift=8pt] at (5,2)
			{\includegraphics[width=3.0cm]{\TLPicThree}};
		\node[anchor=south,inner sep=0,xshift=20pt,yshift=8pt,scale=0.43] at (7,2)
			{\includegraphics[width=3.5cm]{\TLPicFour}};
		\node[anchor=south,inner sep=0,xshift=20pt,yshift=8pt,scale=1.0] at (9.7,2)
			{\includegraphics[width=3.5cm]{\TLPicFive}};

	% Labels:
		\fill[black]
			(-2.5,5.5) 		node[anchor=west,inner sep=0, scale=1.1] 
				{\TLTitle}
			(10.3,2)		node[below right,,xshift=-20pt,yshift=-5pt,scale=.9,text width=2.5cm,align=left]
				{\TLTextFive}
			(7.5,2) 		node[below right,,xshift=-20pt,yshift=-5pt,scale=.9,text width=2.5cm,align=left]
				{\TLTextFour}
			(5.1,1.9) 		node[below right,xshift=-20pt,scale=.9,text width=2cm,align=left]
				{\TLTextThree}
			(1.9,1.9) 		node[below right,xshift=-10pt,scale=.9,text width=2cm,align=left]
				{\TLTextTwo}
			(-1.4,2) 		node[below right,xshift=-20pt,yshift=-5pt,scale=0.9,text width=2.8cm,align=left]
				{\TLTextOne} 
		;
	\end{scope} 
\end{tikzpicture}
\end{document}